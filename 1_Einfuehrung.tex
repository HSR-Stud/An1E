
%%%%%%%%%%%%%%%%%%%%%%%%%%%%%%%%%%%%%%%%%%%%%%%%%%%%%%%%%%%%%%%%%%%%%%%%%%%%%%%%%%%%%%%%%%%%%%%%
% Einführung                                   
%%%%%%%%%%%%%%%%%%%%%%%%%%%%%%%%%%%%%%%%%%%%%%%%%%%%%%%%%%%%%%%%%%%%%%%%%%%%%%%%%%%%%%%%%%%%%%%%
\section{Einf"uhrung}
\subsection{Zahlenmengen\formelbuchgreen{1,331}}
	\begin{minipage}[c]{6.5cm}
		$ \mathbb{N}  = \left\{1,2,3,...\right\};\; $\\
		$ \mathbb{Q} = \left\{x|x \;=\; ^{p}/_{q} \text{ mit } p \in \mathbb{Z}
		\text{ und } (q \in \mathbb{Z} \smallsetminus \{0\})\right\};\;$
	\end{minipage}
	\begin{minipage}[c]{5cm}
		$ \mathbb{N}_0  = \left\{0,1,2,3,...\right\};\; $\\
	\end{minipage}
	\begin{minipage}[c]{5cm}
		$ \mathbb{Z} = \left\{...,-2,-1,0,1,2,..\right\}; $\\
		$ \mathbb{R} = zB \; \sqrt{2}, \pi,\phi$
	\end{minipage}

\subsection{Mengenlehre\formelbuchgreen{334}}
	$A \;=\; \left\{-2,-1,0,1,2\right\} ,\; B\; =\; \left\{0,1,2,3,4\right\}$\\
	\begin{minipage}[c]{6.5cm}
		Schnittmenge:\\
		Vereinigungsmenge:\\
		Differenzmenge:\\
		Produktmenge:\\
		Kommutativgesetz:\\
		Assoziativgesetz:\\
		Distributivgesetz:
	\end{minipage}
	\begin{minipage}[c]{6.5cm}
		$A \; \cap  B \;=\;\left\{x|x \in A \text{ und } x \in B \right\}$\\
		$A \; \cup  B \;=\;\left\{x|x \in A \text{ oder } x \in B \right\}$\\			
		$A \; \smallsetminus B \;=\;\left\{x|x \in A \text{ und } x \notin B \right\}$\\
		$A \; \times B\;=\;\left\{(a,b)|a \in A \text{ und } b \in B \right\}$\\
		$A \; \cap  B \;=\;B \; \cap A$ \\
		$\left(A \cap  B \right) \cap C\;=\;A \cap \left( B \cap C \right)$ \\
		$A\; \cap \left(B\cup C\right)\;=\;\left(A \cup B\right)\cap \left(A \cup C\right) $ 
	\end{minipage}
	\begin{minipage}[c]{7cm}
		$A \; \cap B \;=\; \left\{0,1,2\right\}$\\
		$A \; \cup  B \;=\;\left\{-2,-1,0,1,2\right\}$\\
		$A \; \smallsetminus B \;=\;\left\{-2,-1\right\}$\\
		$ $\\
		$A \; \cup  B \;=\;B \; \cup A$ \\
		$\left(A \cup  B\right) \cup C\;=\;A \cup \left( B \cup C \right)$ \\
		$A\; \cup \left( B \cap C \right)\;=\;\left( A \cap  B \right) \cup \left(A \cap  C\right) $
	\end{minipage}
	
\subsection{Beweismethoden\formelbuchgreen{5}}

\subsection{Spezielle Ungleichungen\formelbuchgreen{30}}
	\begin{tabbing}	xxxxxxxxxxxxxxxxxxxxxxxxxxxxxxx\=xxxxxxxxxxxxxxxxxxxxxx\=xxxxxxxxxxxxxxxxxxxxxx\=xxxxxxxxxxxxxxxxxxxxxx\=\kill
		Bernoulli-Ungleichung: \>
			$(1 + a)^n > 1 + n \cdot a$\>
				f"ur $n \in N, n \geq 2, a \in R, a > -1, a\neq0$\\
		Binomische Ungleichung: \>
			$|a\cdot b|\leq\frac{1}{2}(a^2 + b^2)$\\
		Dreiecksungleichung: \>
			$\left|a+b\right|\leq\left|a\right|+\left|b\right|$ \>
				$\left|a-b\right|\leq\left|a\right|+\left|b\right|$ \>
					$\left|a-b\right|\geq\left|\left|a\right|-\left|b\right|\right|$\\
		Geometrisches und arithmetisches Mittel:\\ 
		f"ur $a_i\geq0,\;n \in \mathbb{N},\;i \in \left\{1,2,...,n \right\}:$\>
			$\sqrt[n]{a_1 a_2 \ldots a_n}\leq \frac{1}{n} \cdot \sum\limits _{i=1}^n a_i = 	\frac{a_1+a_2+...+a_n}{n}$\>\>
			$\sqrt{ab}\leq \frac{a+b}{2}$, siehe Br. S.19/20 \\
		Minima/Maxima: \>
			$\min\{a_i\} \leq \sqrt[n]{a_1a_2 \ldots a_n} \leq \max\{a_i\}$\\
		Betragsungleichung:\>$-c<x<c\;\Leftrightarrow\;|x|<c$
	\end{tabbing}

\subsection{Umgebung}
	\begin{minipage}[c]{14.5cm}
		Jedes offene Intervall, dass die Zahl a enth"alt, heisst eine Umgebung von a. \\
		Es sei $\epsilon >$ 0. Unter der $\epsilon$-Umgebung von a versteht man das offene Intervall $(a-\epsilon,a+\epsilon).$\\
		Eine $\epsilon$-Umgebung von a ohne die Zahl a selbst wird punktierte $\epsilon$-Umgebung von a genannt.
	\end{minipage}
	\begin{minipage}[c]{5cm}
	Schreibweise: U(a)\\
	Schreibweise: $U_\epsilon(a)$\\
	Schreibweise: $\dot{U}_\epsilon(a)=U_\epsilon(a)\smallsetminus{a}$
	\end{minipage}

\subsection{Summenzeichen\formelbuchgreen{7}}
	\begin{minipage}[c]{4.75cm}
		$\text{mit 1}\leq m\leq n $
	\end{minipage}
	\begin{minipage}[c]{16cm}
			Die Laufvariable $i$ wird immer um 1 aufaddiert. $i$ immer kleiner-gleich $n$ (z.B. wenn $i \in \mathbb{R}$)
	\end{minipage}
	\begin{minipage}[c]{4.75cm}
		$\sum\limits _{i=1}^n a_i = \sum\limits _{i=1}^m a_i + \sum\limits _{i=m+1}^n a_i;$
	\end{minipage}
	\begin{minipage}[c]{4.25cm}	
		$\sum\limits _{i=1}^n a_i = \sum\limits _{i=1-j}^{n-j} a_{i+j};$
	\end{minipage}
	\begin{minipage}[c]{4.25cm}
		$\sum\limits _{i=1}^n a = n\cdot a;$
	\end{minipage}
	\begin{minipage}[c]{8cm}
		$\sum\limits _{i=1}^n \left(\lambda a_i + \beta b_i \right) = $
		$\lambda \sum\limits _{i=1}^n a_i + \beta \sum\limits _{i=1}^n b_i$	
	\end{minipage}
	
\subsection{Spezielle endliche Reihen\formelbuchgreen{19}}
			\begin{minipage}[c]{4.25cm}
				$\sum\limits _{i=1}^n i = \frac{n(n+1)}{2}$
			\end{minipage}
			\begin{minipage}[c]{4.25cm}	
				$\sum\limits _{i=1}^n i^2 = \frac{n(n+1)(2n+1)}{6}$
			\end{minipage}
			\begin{minipage}[c]{4.25cm}
				$\sum\limits _{i=1}^n i^3 = \frac{n^2(n+1)^2}{4}$
			\end{minipage}

\subsection{Produktzeichen\formelbuchgreen{7}}
	$a_n\prod\limits _{i=1}^n \left(x-x_i\right)=
	a_n\cdot\left(x-x_1\right)\cdot\left(x-x_2\right)\cdot...\cdot\left(x-x_n\right)$
	
\subsection{Fakult"at\formelbuchgreen{13}}
	\begin{minipage}[c]{6cm}
		$n! = 1\cdot2\cdot3\cdot...\cdot n $
	\end{minipage}
	\begin{minipage}[c]{6cm}
		$\text{f"ur n} \in \mathbb{N}, n \geq 3$
	\end{minipage}
	\begin{minipage}[c]{6cm}
		$n!>2^{n-1}$
	\end{minipage}
	
\subsection{Binomischer Satz\formelbuchgreen{12}}
	\begin{minipage}[c]{6cm}
		$\left(a+b\right)^n = \sum\limits _{i=0}^n \left(\stackrel{n}{i}\right)a^{n-i}\cdot b^i$\\
		$\left(\stackrel{n}{i-1}\right)+\left(\stackrel{n}{i}\right)=\left(\stackrel{n+1}{i}\right)$
	\end{minipage}
	\begin{minipage}[c]{6cm}
		$\left(\stackrel{n}{i}\right)=\left(\stackrel{n}{n-i}\right)$\\
		$\left(\stackrel{n}{i}\right)=\frac{n!}{i!\left(n-i\right)!}$
	\end{minipage}
	\begin{minipage}[c]{6cm}	
		$\left(\stackrel{n}{0}\right)=1$\\
		$2^n = \sum\limits _{i=0}^n \left(\stackrel{n}{i}\right)$
	\end{minipage}
	
\subsection{Einige Wurzeln}
$\sqrt{2} = 1.414; \qquad \sqrt{3} = 1.732; \qquad \sqrt{5} = 2.236; \qquad \sqrt{6} = 2.449; \qquad \sqrt{7} = 2.645; \qquad \sqrt{8} = 2.828;$