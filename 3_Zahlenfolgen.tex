%%%%%%%%%%%%%%%%%%%%%%%%%%%%%%%%%%%%%%%%%%%%%%%%%%%%%%%%%%%%%%%%%%%%%%%%%%%%%%%%%%%%%%%%%%%%%%%%%
%% Zahlenfolgen und Grenzwerte                  
%%%%%%%%%%%%%%%%%%%%%%%%%%%%%%%%%%%%%%%%%%%%%%%%%%%%%%%%%%%%%%%%%%%%%%%%%%%%%%%%%%%%%%%%%%%%%%%%%
\section{Zahlenfolgen\formelbuchorange{18,469}}
\subsection{Einf"uhrung\formelbuchorange{18}}

\begin{minipage}[c]{5cm}
	arithmetische Folge: \\
	$a_1=c$ und $a_{n+1}=a_n+d$ \\
	\\
	geometrische Folge:\\
	$a_1=c$ und $a_{n+1}=q*a_n$\\
\end{minipage}
\begin{minipage}[c]{2.5cm}
	$ $\\
	$\underbrace{d=a_{n+1}-a_n}_{Differenz}$\\
	$ $\\
	$\underbrace{q=\frac{a_{n+1}}{a_n}}_{Quotient}$
\end{minipage}
\begin{minipage}[c]{10cm}	
	\begin{tabular}{|c|c|c|c|}
		\hline
		\multicolumn{4}{|c|}{Monotonie} \\
		\hline
		$d \geq 0$ & $q \geq 1$ & monoton wachsend & $\uparrow$\\
		\hline
		$d > 0$ & $q > 1$ & streng monoton wachsend & $\Uparrow$\\
		\hline
		$d \leq 0$  & $0 < q \leq 1$ & monoton fallend & $\downarrow$\\
		\hline
		$d < 0$ & $0 < q < 1$ & streng monoton fallend & $\Downarrow$\\
		\hline
	\end{tabular}
\end{minipage}
	\\
	konstante Folge:\\
	$a_1=c$ und $a_{n+1}=c$

\subsection{Beschr"anktheit\formelbuchorange{51,469}}
	$\text{Beschr"ankt wenn }k\leq a_n\leq\text{ K,	wobei k bzw. K die untere bzw. obere Schranke
	ist}$
\subsubsection{Bolzano-Weierstrass}
	Jede beschr"ankte und monotone Zahlenfolge ist konvergent.

\subsection{Grenzwerts"atze\formelbuchorange{470}}

\subsection{Grenzwerte von rekursiven Folgen}

\begin{flushleft}
\begin{enumerate}
	\item Hypothetischer Grenzwert ausrechnen \\[5pt]
		\begin{minipage}[t]{4.5cm}
			$  \lim\limits_{n\rightarrow\infty} a_n = \lim\limits_{n\rightarrow\infty} a_{n+1} = a $
			\end{minipage}
			\begin{minipage}[t]{12cm}
			$\text{z.B.}\sqrt{a_n}+1=a_{n+1}\Rightarrow\lim\limits_{n\rightarrow\infty}\;\sqrt{a}+1 = a$\\
			$\Rightarrow x=\frac{3\pm\sqrt{5}}{2}\Rightarrow$ M"oglicher Grenzwert $\rightarrow$ wenn Folge beschr"ankt und monoton 
			\end{minipage}\\[5pt]	

	\item Beschr"anktheit mittels des hypothetischen Grenzwertes\\
$\Rightarrow$ mit vollst"andider Induktion beweisen (Auch mit Ungleichungen l"osbar)\\
			\begin{minipage}[t]{20cm}
				\begin{minipage}[t]{1cm}
					z.B.
				\end{minipage}
				\begin{minipage}[t]{5cm}
					Induktionsanfang $A(1)$:\\
					Induktionsschritt $A(n+1)$:\\
				\end{minipage}
				\begin{minipage}[t]{15cm}
					$a_1 < \frac{3+\sqrt{5}}{2} < 3 \Rightarrow 1 < 3 $\\
					$a_n < 3$ \hspace{5cm}$|\sqrt{...}\quad |+1$\\
					$\underbrace{\sqrt{a_n}+1}_{a_{n+1}} < \sqrt{3}+1 < 3$\\
				\end{minipage}   
			\end{minipage}  



	\item Monotonie annehmen (ev. erste Glieder berechnen)\\
		$\Rightarrow$ mit vollst"andiger Induktion beweisen (Auch mit q/d-Kriterium oder Ungleichungen l"osbar)\\
			\begin{minipage}[t]{20cm}
				\begin{minipage}[t]{1cm}
					z.B.
				\end{minipage}
				\begin{minipage}[t]{5cm}
					Induktionsanfang $A(1)$:\\
					Induktionsschritt $A(n+1)$:\\
				\end{minipage}
				\begin{minipage}[t]{15cm}
					$a_1 < a_2 \Rightarrow 1 < 2 $\\
					$a_n < a_{n+1}$ \hspace{5cm}$|\sqrt{...}\quad |+1$\\
					$\underbrace{\sqrt{a_n}+1}_{a_{n+1}} < \underbrace{\sqrt{a_{n+1}}+1}_{a_{n+2}}$\\
				\end{minipage}   

			\end{minipage}  
            \item Grenzwert bestimmen\\
		$\Rightarrow$ Aus 2. und 3. folgt das $x=\frac{3+\sqrt{5}}{2}\rightarrow$ Da Folge nach oben beschr"ankt und streng monoton wachsend
\end{enumerate}
					



\subsection{$\varepsilon - n_0$ - Kriterium\formelbuchorange{470}}
	$|a_n - a| < \varepsilon$ f"ur alle $n \geq n_0(\varepsilon)$

\end{flushleft}
